\chapter{Analysis}
\label{chap:Analysis}

\section{Introduction}
In this chapter an analysis of the problem is presented, along with an outline and justification of what will be implemented.

\section{The Problem}

For MDE tools to be seriously considered for large scale projects, these tools need to have a comprehensive set of features. The area that I have chosen to work on is test coverage in Epsilon.

In the literature review, the section titled `Quality of Software Testing' discussed various approaches to determining how thorough a test set is. Based on that research, I will begin by attempting to implement a statement coverage tool for Epsilon. While statement coverage can be considered as a weak measure of coverage, it is simple to implement and can quickly inform developers if there is any code that has gone untested in at least one way.

%Rather than implement statement analysis for just the EuGENia test suite, I will implement it for EOL in general, and attempt to detail a general approach for statement analysis in any language. As Epsilon has many languages, my documentation may prove useful for a future developer who wishes to implement statement analysis in another Epsilon language, or even a non-Epsilon language that does not yet have a statement analysis tool implemented.

In the literature review I explained that Epsilon has a great number of languages, all for different purposes. To implement statement coverage for all of these languages would be tedious and not particularly interesting. EOL is the base of all of the other languages, and so if I implement and test it for EOL, a future project could continue my work by extending statement coverage to work for all languages.

In the book by \citet{Myers:2004:AST:983238}, they state that statement analysis is the easiest form of coverage analysis, but is not particularly useful. Once a developer has a high level of statement coverage, it would be useful for them if they had a stronger form of analysis that could be performed. There are a few options, but branch analysis seems to offer a good trade-off between usefulness and difficulty of implementation. Once again branch analysis will only be implemented for the EOL language, with the idea that a future project could expand it to also cover all other Epsilon languages.

Again, \citet{Myers:2004:AST:983238} are critical of branch analysis, and suggest that path coverage is a better metric of test quality. However, it is more difficult to implement than branch analysis. I will aim to implement path analysis in Epsilon for EOL files only, for the same reasons as statement and branch analysis.

Mutation testing is a strong form of testing, but is difficult to implement and I would argue that it would be not as useful to developers as statement, branch and path coverage. With this in mind, I have decided to not develop mutation testing for EOL at this time. 

%If time permits then I will attempt to implement mutation testing. However, research suggests that this is an incredibly difficult problem that may take longer to solve than the time that I have available, and as a compromise I may investigate how it could be implemented in EOL, and suggest the way forward for a future project.

For each of the above forms of analysis I have not yet specified how exactly the coverage metric will be presented to the user. In the literature review I looked at some software packages that were plugins to Eclipse that highlighted statements and branches that were covered or not covered. If time permits then I will look into creating an Eclipse plugin. However, the main focus of my project is to actually produce various coverage metrics, and so most of my time will be focused on getting those values. If I do not have much time towards the end of the project, then I will simply output a text or HTML file of some sort that details the results of the analysis.

It is assumed that the three forms of analysis (statement, branch and path) will each be dependent on the previous. So branch analysis cannot be completed until statement analysis has been completed. While this is not strictly true, it is likely that in practice that this will be the case. As I will be new to the Epsilon source code, it makes sense to implement the easiest form of coverage first.