\section{Testing}

As with statement coverage, the solution must be tested. Each of the basic statements will be tested individually, and then some combinations of statements will be tested. Obviously there are an infinite number of input combinations, and only a small subsection of these can be tested. I will attempt to choose interesting inputs that require difficult to produce CFGs, rather than something less interesting, such as 10 sequential if statements. Automatic verification of the produced CFGs would be very difficult to perform, and so all CFGs will be checked by hand. This of course means that there is a risk of human error, but at this moment I can see no way to alleviate this risk.

\subsection{Individual Statements}

The general criteria for a test to pass are as follows:

\begin{itemize}[nolistsep]
\item There is a START vertex at the top of the CFG which has no parents, and at least one child.
\item There is an END vertex that has at least one parent, and no children.
\item The produced CFG matches the one that was shown in the analysis section for that statement.
\item Other than the end vertex, all vertices have at least one child
\end{itemize}

Each of the statements from the list in the analysis section will be tested. Because of space limitations for loops and while loops are considered the same (because they use the same code).

\begin{figure}
\centering
\begin{minipage}[b]{.25\textwidth}
  \centering
  \includedot[scale=0.3]{figures/branchtest/block}
  \caption{}
  \label{fig:testBlock}
\end{minipage}%
\begin{minipage}[b]{.25\textwidth}
  \centering
  \includedot[scale=0.3]{figures/branchtest/if}
  \caption{}
  \label{fig:testIf}
\end{minipage}%
\begin{minipage}[b]{.25\textwidth}
  \centering
    \includedot[scale=0.3]{figures/branchtest/ifElse}
    \caption{}
  	\label{fig:testIfElse}
\end{minipage}
\begin{minipage}[b]{.24\textwidth}
  \centering
  \includedot[scale=0.3]{figures/branchtest/for}
  \caption{}
  \label{fig:testFor}
\end{minipage}
\end{figure}

Testing begins with the block vertex. This will be done by having a single line of code in a program, with no conditional statements.

\begin{samepage}
\begin{minipage}{.6\textwidth}
\begin{description}[style=sameline,leftmargin=4.5cm,nolistsep]
\item[\hspace*{0.3cm}Label] BT-01
\item[\hspace*{0.3cm}Statement under Test] BLOCK
\item[\hspace*{0.3cm}Expected Output] See Figure \ref{fig:block}
\item[\hspace*{0.3cm}Result] Pass (see Figure \ref{fig:testBlock})
\end{description}
\end{minipage}
\begin{minipage}{.1\textwidth}
\hspace{1.0mm}
\end{minipage}
\begin{minipage}{.29\textwidth}
  \centering
  Code used for testing:
  \lstinputlisting[language=EOL]{code/branchtest/block.eol}
\end{minipage}
\end{samepage}

Of note about this test is that the operation call \verb|println| is printed, but not the string that is to be printed. This is because the type string is not on the whitelist, but the operation call type is. If both were on the whitelist, then two vertices would be used to represent that line of code.

\begin{samepage}
\begin{minipage}{.6\textwidth}
\begin{description}[style=sameline,leftmargin=4.5cm,nolistsep]
\item[\hspace*{0.3cm}Label] BT-02
\item[\hspace*{0.3cm}Statement under Test] if
\item[\hspace*{0.3cm}Expected Output] See Figure \ref{fig:if}
\item[\hspace*{0.3cm}Result] Pass (see Figure \ref{fig:testIf})
\end{description}
\end{minipage}
\begin{minipage}{.1\textwidth}
\hspace{1.0mm}
\end{minipage}
\begin{minipage}{.29\textwidth}
  \centering
  Code used for testing:
  \lstinputlisting[language=EOL]{code/branchtest/if.eol}
\end{minipage}
\end{samepage}

The if statement does not require any code in its block for this test, because leaving the space between the parenthesis empty produces a block vertex, which is enough for this test. The if statement above will always execute because its parameter is \verb|true|, but again for testing this is not relevant because the generated CFG does not take this into account.

\begin{samepage}
\begin{minipage}{.6\textwidth}
\begin{description}[style=sameline,leftmargin=4.5cm,nolistsep]
\item[\hspace*{0.3cm}Label] BT-03
\item[\hspace*{0.3cm}Statement under Test] if .. else
\item[\hspace*{0.3cm}Expected Output] See Figure \ref{fig:ifelse}
\item[\hspace*{0.3cm}Result] Pass (see Figure \ref{fig:testIfElse})
\end{description}
\end{minipage}
\begin{minipage}{.1\textwidth}
\hspace{1.0mm}
\end{minipage}
\begin{minipage}{.29\textwidth}
  \centering
  Code used for testing:
  \lstinputlisting[language=EOL]{code/branchtest/ifElse.eol}
\end{minipage}
\end{samepage}

\begin{samepage}
\begin{minipage}{.6\textwidth}
\begin{description}[style=sameline,leftmargin=4.5cm,nolistsep]
\item[\hspace*{0.3cm}Label] BT-04
\item[\hspace*{0.3cm}Statement under Test] for
\item[\hspace*{0.3cm}Expected Output] See Figure \ref{fig:for}
\item[\hspace*{0.3cm}Result] Pass (see Figure \ref{fig:testFor})
\end{description}
\end{minipage}
\begin{minipage}{.1\textwidth}
\hspace{1.0mm}
\end{minipage}
\begin{minipage}{.29\textwidth}
  \centering
  Code used for testing:
  \lstinputlisting[language=EOL]{code/branchtest/for.eol}
\end{minipage}
\end{samepage}

\begin{figure}
\centering
\begin{minipage}{.25\textwidth}
  \centering
  \includedot[scale=0.3]{figures/branchtest/while}
  \caption{}
  \label{fig:testWhile}
\end{minipage}%
\begin{minipage}{.25\textwidth}
  \centering
  \includedot[scale=0.3]{figures/branchtest/switch1}
  \caption{}
  \label{fig:testSwitch1}
\end{minipage}%
\begin{minipage}{.25\textwidth}
  \centering
    \includedot[scale=0.3]{figures/branchtest/switch2}
    \caption{}
  	\label{fig:testSwitch2}
\end{minipage}
\begin{minipage}{.24\textwidth}
  \centering
  \includedot[scale=0.3]{figures/branchtest/operation}
  \caption{}
  \label{fig:testOperation}
\end{minipage}
\end{figure}

\begin{minipage}{.6\textwidth}
\begin{description}[style=sameline,leftmargin=4.5cm,nolistsep]
\item[\hspace*{0.3cm}Label] BT-05
\item[\hspace*{0.3cm}Statement under Test] switch (with fallthrough, without a default statement)
\item[\hspace*{0.3cm}Expected Output] See Figure \ref{fig:switch1}
\item[\hspace*{0.3cm}Result] Pass (see Figure \ref{fig:testSwitch1})
\end{description}
\end{minipage}
\begin{minipage}{.1\textwidth}
\hspace{1.0mm}
\end{minipage}
\begin{minipage}{.29\textwidth}
  \centering
  Code used for testing:
  % This line breaks the document for some reason
  \lstinputlisting[language=EOL,breaklines=true]{code/branchtest/switch1.eol}
  %\lstinputlisting{code/branchtest/switch1.eol}
\end{minipage}

\begin{minipage}{.6\textwidth}
\begin{description}[style=sameline,leftmargin=4.5cm,nolistsep]
\item[\hspace*{0.3cm}Label] BT-06
\item[\hspace*{0.3cm}Statement under Test] switch (without fallthrough, with a default statement)
\item[\hspace*{0.3cm}Expected Output] See Figure \ref{fig:switch2}
\item[\hspace*{0.3cm}Result] Pass (see Figure \ref{fig:testSwitch2})
\end{description}
\end{minipage}
\begin{minipage}{.1\textwidth}
\hspace{1.0mm}
\end{minipage}
\begin{minipage}{.29\textwidth}
  \centering
  Code used for testing:
  \lstinputlisting[language=EOL]{code/branchtest/switch2.eol}
\end{minipage}

Both context-free and contextual operations were tested in the following test. Using the example from The Epsilon Book \citep{epsilonBook} one operation of each type was defined, and then both operations called.

\begin{minipage}{.6\textwidth}
\begin{description}[style=sameline,leftmargin=4.5cm,nolistsep]
\item[\hspace*{0.3cm}Label] BT-07
\item[\hspace*{0.3cm}Statement under Test] Operation call
\item[\hspace*{0.3cm}Expected Output] See Figure \ref{fig:testOperation}
\item[\hspace*{0.3cm}Result] Pass (see right-hand CFG of Figure \ref{fig:operationOptions}). See below.
\end{description}
\end{minipage}
\begin{minipage}{.1\textwidth}
\hspace{1.0mm}
\end{minipage}
\begin{minipage}{.29\textwidth}
  \centering
  Code used for testing:
  \lstinputlisting[language=EOL,breaklines=true]{code/branchtest/operation.eol}
\end{minipage}

Notice that the operations are actually shown in the wrong order in the generated CFG. \verb|add1| would be executed first, because it is passed as a parameter to \verb|add2|. Looking at the AST for this code it is easy to see why. \verb|add1| is a child of \verb|add2|, because it is a parameter. When a depth-first traversal is performed, \verb|add2| is found before \verb|add1|, and so it is added to the CFG first. While this is technically incorrect, for purposes of calculating branch coverage it does not matter. However, for the more general purpose use, the operation call function would need to traverse its children to find any other operation call vertices, and add them to the CFG in reverse order. This could be done easily by recursively calling a function on the children of the AST vertex, and after the recursion checking to see if the type is an operation call. If so, then add it to the CFG and update the global last pointer to that vertex.

\begin{figure}
\centering
\begin{minipage}{.25\textwidth}
  \centering
  \includedot[scale=0.3]{figures/branchtest/break}
  \caption{}
  \label{fig:testBreak}
\end{minipage}%
\begin{minipage}{.25\textwidth}
  \centering
  \includedot[scale=0.3]{figures/branchtest/breakAll}
  \caption{}
  \label{fig:testBreakAll}
\end{minipage}%
\begin{minipage}{.25\textwidth}
  \centering
    \includedot[scale=0.3]{figures/branchtest/continue}
    \caption{}
  	\label{fig:testContinue}
\end{minipage}
\end{figure}
For the break, breakAll and continue statements, they were placed in an if statement so that the control flow was also created for when they are not executed. In separate testing that is not documented due to space constraints, they were also tested on their own and proved to work as expected.

\begin{minipage}{.6\textwidth}
\begin{description}[style=sameline,leftmargin=4.5cm,nolistsep]
\item[\hspace*{0.3cm}Label] BT-08
\item[\hspace*{0.3cm}Statement under Test] Break
\item[\hspace*{0.3cm}Expected Output] An edge from break to while.
\item[\hspace*{0.3cm}Result] Pass (see right-hand CFG of Figure \ref{fig:testBreak}). See below.
\end{description}
\end{minipage}
\begin{minipage}{.1\textwidth}
\hspace{1.0mm}
\end{minipage}
\begin{minipage}{.29\textwidth}
  \centering
  Code used for testing:
  \lstinputlisting[language=EOL,breaklines=true]{code/branchtest/break.eol}
\end{minipage}

For break, a nested while loop was used to prove that the break statement only breaks out of one while loop, unlike the breakAll statement.

\begin{minipage}{.6\textwidth}
\begin{description}[style=sameline,leftmargin=4.5cm,nolistsep]
\item[\hspace*{0.3cm}Label] BT-09
\item[\hspace*{0.3cm}Statement under Test] BreakAll
\item[\hspace*{0.3cm}Expected Output] An edge from breakAll to END.
\item[\hspace*{0.3cm}Result] Pass (see Figure \ref{fig:testBreakAll}). See below.
\end{description}
\end{minipage}
\begin{minipage}{.1\textwidth}
\hspace{1.0mm}
\end{minipage}
\begin{minipage}{.29\textwidth}
  \centering
  Code used for testing:
  \lstinputlisting[language=EOL,breaklines=true]{code/branchtest/breakAll.eol}
\end{minipage}

For breakAll, two while loops were used to prove that the breakAll statement can break out of more than just the one loop that the break statement does.

\begin{minipage}{.6\textwidth}
\begin{description}[style=sameline,leftmargin=4.5cm,nolistsep]
\item[\hspace*{0.3cm}Label] BT-10
\item[\hspace*{0.3cm}Statement under Test] Continue
\item[\hspace*{0.3cm}Expected Output] An edge from continue to while.
\item[\hspace*{0.3cm}Result] Pass (see Figure \ref{fig:testContinue}). See below.
\end{description}
\end{minipage}
\begin{minipage}{.1\textwidth}
\hspace{1.0mm}
\end{minipage}
\begin{minipage}{.29\textwidth}
  \centering
  Code used for testing:
  \lstinputlisting[language=EOL,breaklines=true]{code/branchtest/continue.eol}
\end{minipage}

\subsection{Multiple Statements Testing}

During the previous section I tested that individual statements are handled correctly. Some of the tests also had multiple statements in, such as the break statement. Now that this has been proven, a bigger test case will be created and run. During development of course many large blocks of code were tested, but once again due to the space constraints in this document I can only record one of them. The even bigger test will be when EuGENia is used as a case study, but that is more about testing how well the coverage metrics work, rather than how well the AST to CFG conversion works.

This test has been designed to use most of the statements that are covered in the analysis section. The code is highly contrived, and if code like this were seen in a real application then you would probably want to question the programmer's technique. However, that doesn't mean that my conversion algorithm shouldn't be able to handle it, and so in Figure \ref{fig:testComplex} is the generated CFG on the left, and the code used on the right.

A manual walkthrough of the CFG confirms that it does as is expected. The only point of interest is at the bottom right, where there are two vertices labelled `i'. This maps to the \verb|i = i + 1;| statement on line 8 of the code. Analysis of the AST using the AST Explorer shows that the vertex i is of type Feature Call, which is the same as is used for operation calls. I could remove operation calls from the CFG whitelist, but I think that it is useful to have them in the CFG, and so I will leave it as is.

\begin{figure}
\begin{minipage}{.6\textwidth}
\centering
	\includedot[scale=0.35]{figures/branchtest/complex1}
\end{minipage}
\begin{minipage}{.39\textwidth}
  \centering
  Code used for testing:
  \lstinputlisting[language=EOL]{code/branchtest/complex1.eol}
\end{minipage}
\caption{The complex test code and the generated CFG}
\label{fig:testComplex}
\end{figure}


\subsection{Branch Coverage Testing}

Now that I am satisfied that the CFG is correctly, it can be used to perform branch analysis.

\begin{minipage}{0.6\textwidth}
\begin{description}[style=sameline,leftmargin=3.5cm,nolistsep]
\item[\hspace*{0.3cm}Label] BT-11
\item[\hspace*{0.3cm}Description] An if .. else statement that just executes the if's true block
\item[\hspace*{0.3cm}Expected Output] if .. else CFG with the edge from the if vertex to the true block vertex coloured in green, and the other side in red.
\item[\hspace*{0.3cm}Result] Pass
\end{description}
\end{minipage}
\begin{minipage}{0.39\textwidth}
\centering
\includedot[scale=0.3]{figures/branchtest/colIf}
\end{minipage}

\begin{minipage}{0.6\textwidth}
\begin{description}[style=sameline,leftmargin=3.5cm,nolistsep]
\item[\hspace*{0.3cm}Label] BT-12
\item[\hspace*{0.3cm}Description] An if .. else statement that just executes the if's false block
\item[\hspace*{0.3cm}Expected Output] if .. else CFG with the edge from the if vertex to the true block vertex coloured in red, and the other side in green.
\item[\hspace*{0.3cm}Result] Pass
\end{description}
\end{minipage}
\begin{minipage}{0.39\textwidth}
\centering
\includedot[scale=0.3]{figures/branchtest/colIfFalse}
\end{minipage}

\begin{minipage}{0.6\textwidth}
\begin{description}[style=sameline,leftmargin=3.5cm,nolistsep]
\item[\hspace*{0.3cm}Label] BT-13
\item[\hspace*{0.3cm}Description] A switch statement that only executed the default block
\item[\hspace*{0.3cm}Expected Output] Switch CFG with the edge between the switch vertex and the default vertex coloured in green. 
\item[\hspace*{0.3cm}Result] Fail
\end{description}
\end{minipage}
\begin{minipage}{0.39\textwidth}
\centering
\includedot[scale=0.3]{figures/branchtest/switchFail}
\end{minipage}

Test BT-13 has failed because the default branch was not coloured in green, meaning that it has not been marked as executed. After some investigation, the bug was found and fixed. When a switch statement is executed, the first child of each case statement is executed. This initially caused every case statement to be marked as executed, and so an exception was added to prevent this. Unfortunately this code then broke the code that marked the default branch as being executed. A fix was implemented, and the switch statement thoroughly tested, as shown in Figure \ref{fig:testSwitch}

\begin{figure}
\begin{minipage}{0.19\textwidth}
\includedot[scale=0.3]{figures/branchtest/switch1}
\end{minipage}
\begin{minipage}{0.19\textwidth}
\includedot[scale=0.3]{figures/branchtest/switch2}
\end{minipage}
\begin{minipage}{0.19\textwidth}
\includedot[scale=0.3]{figures/branchtest/switch3}
\end{minipage}
\begin{minipage}{0.19\textwidth}
\includedot[scale=0.3]{figures/branchtest/switch4}
\end{minipage}
\begin{minipage}{0.19\textwidth}
\includedot[scale=0.3]{figures/branchtest/switch5}
\end{minipage}
\caption{The thoroughly tested switch statement}
\label{fig:testSwitch}
\end{figure}

\begin{description}[style=sameline,leftmargin=3.5cm,nolistsep]
\item[\hspace*{0.3cm}Label] BT-14
\item[\hspace*{0.3cm}Description] A while loop that contains a break statement that will be executed.
\item[\hspace*{0.3cm}Expected Output] The edge into the while block should be green, but because the break statement is called the other edge from the while vertex will be red.
\item[\hspace*{0.3cm}Result] Pass
\end{description}

\begin{samepage}
\begin{description}[style=sameline,leftmargin=3.5cm,nolistsep]
\item[\hspace*{0.3cm}Label] BT-15
\item[\hspace*{0.3cm}Description] A while loop that contains a break statement that will be executed.
\item[\hspace*{0.3cm}Expected Output] The edge into the while block should be green, but because the break statement is called the other edge from the while vertex will be red.
\item[\hspace*{0.3cm}Result] Pass
\end{description}
\end{samepage}

Many more tests were carried out, but unfortunately due to space requirements they cannot be detailed here.