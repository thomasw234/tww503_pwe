\chapter{Conclusions}
\label{chap:conclusions}

Calculating code coverage for unit tests is not a new idea. This project has made two primary contributions. The first is a framework for statement and branch coverage for EOL programs which has been tested on a large program written in EOL - EuGENia. This testing showed that EuGENia's existing test suite is insufficient, and highlighted the areas that require further testing. The second is a general algorithm for converting an Abstract Syntax Tree to a Control Flow Graph. The algorithm is tailored for EOL syntax, but the descriptions are abstract enough that the process could be applied to any language with a similar syntax. 

A very high level algorithm was provided by \citet{grune2000modern} in the book `Modern Compiler Design', but this was the only literature that could be found that even mentioned the conversion process. The algorithm given in the book is very high level, and does not provide any level of detail, except for an example about an if statement. In this project I have detailed each statement and explained the considerations that must be made. 

\section{Further Work}

\subsection{Conversion Algorithm}

An attempt has been made to detail the conversion algorithm so that it is applicable to any language with a syntax that uses statements common to most programming languages (if, else, switch etc.). Despite this, it has designed around EOL, and therefore may require some modification before it can be used in other languages. One notable case of this is the switch statement that implements fallthrough in a different way to Java and most other modern object-oriented languages. Therefore before the algorithm could be used for Java, how the switch statement should be represented as a CFG would need to be considered. 

Some statements that are common across most OO languages are not included in EOL. One prime example is the do .. while statement. The CFG of a do .. while statement would be unlike any other loop because flow would immediately go to the block of the loop, and then at the end the evaluation of the conditional statement would take place.

\subsection{Languages}

Epsilon has many languages for performing operations on models. EOL is the core language, upon which the other languages are built atop \citep{epsilonBook}. From the beginning the framework was designed with extensibility in mind, and so it should not be difficult to extend the conversion algorithm to work with the other languages and provide coverage for all Epsilon languages.

To adapt the algorithm to work with non-EOL based languages, all that is required is the ability to perform a traversal of the AST for the program, and the ability to identify different types of statements in the tree.

\subsection{Output}
The implementation of the conversion algorithm currently only outputs DOT files that can be rendered in GraphViz. Likewise for statement coverage, the only available output is a HTML file. From a user-perspective, an Eclipse plugin similar to EclEmma that provides a quick visual overview of coverage would be useful. Currently my code provides no public API for getting coverage information, but it could be easily adapted to have one like JaCoCo \citep{JaCoCo}, and tools could be created that generate more useful outputs.

A start was made on HTML output for branch coverage by modifying the statement coverage approach. At present it highlights statements in green that have had all branches executed, yellow those that have had some but not all of their branches executed, and red those that have had no branches executed. A problem that needs fixing though is that an if statement (without an else statement following it) has two branches in the CFG. One is for when the if predicate evaluates to true and goes into the if statement's block. The other branch bypasses the if block and continues the program. Whatever the outcome of the if statement, one of the two branches will be marked as executed, and so the if statement is always coloured in yellow or green, never red. This is different to EclEmma and the other tools, which coloured in an if statement red if its predicate only ever evaluated to false. This could be fixed by modifying the CFG class to contain a boolean that marked a branch if it was a `bypass' branch, and the HTML generator not counting the bypass branch when it calculated the colour to use.

\subsection{Performance}

While the performance is arguably acceptable (branch coverage makes execution take 23\% longer) because coverage analysis is not going to be performed every time the code is executed, it would be nice if the code ran a bit faster. As I explained in the results section, I believe that the operation to determine whether the statement that is currently being executed is in the same operation as the previously executed statement is causing a large slowdown. This could be confirmed by performing some code profiling, and then an attempt to come up with a more efficient algorithm could be implemented. One possible approach would be to cache the results of the operation in question. This would be fairly trivial to implement, and should improve performance quite significantly. %One possible approach would be for every statement to be marked as being in a particular function before execution begins. At the moment, if a statement is executed a million times, the slow function will be called a million times, which is un-necessary.

\subsection{Programs}

During the evaluation of EuGENia, it was noted that the EuGENia source had to be merged from 3 files into 1 so that the statement and branch coverage would work correctly. A small modification to the statement coverage would allow it to deal with split files. For the branch coverage, it would likely be a more difficult task to fix. I suspect that there may be a way in Epsilon to get the list of ASTs for every parsed file, in which case each AST could be parsed and added to a single CFG. However, this would require some further investigation.

\subsection{Path Coverage}

Path coverage is not too far from being completed. Implementing the AST to CFG conversion took a lot longer than hoped, but because of that it is now a fairly straightforward task to implement path coverage. The biggest challenge will be recording the path that has been taken by the program while bearing in mind that there is no need to record every iteration of a loop, and then comparing it to other classes. Some of the code that was written for branch coverage to determine whether or not a branch has been executed will be very useful in completing path coverage.

\section{Reflection}

The lack of a detailed algorithm for the conversion from AST to CFG was a surprise. The existing coverage tools that were found did not implement path coverage, which could be explained by the difficult task of building a CFG from an AST.

I hope that the contribution to Epsilon that this project provides will help to justify the case for using Model Driven Engineering for large projects. It is not the only feature that is required, again going back to Martin Fowler's comment about the lack of source control \cite{fowlerMDE}. However, it is definitely a step in the right direction.

I have enjoyed this project and learned a lot about model driven engineering and test coverage. Designing the conversion algorithm was an incredibly difficult challenge, and rather infuriating at times. It took a lot longer to implement the conversion than I had hoped, and I am disappointed that I did not manage to complete the path analysis section. I will complete the path analysis section in my own time, and make it available from the Git repository (\url{https://github.com/thomasw234/EpsilonCodeCoverage}), where the rest of the code is available.