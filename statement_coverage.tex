\chapter{Statement Coverage}

\section{Introduction}

This chapter details my effort to implement statement coverage for EOL programs. I begin by analysing the Epsilon source code that I will be working with. I then move on to detailing the design and implementation of the solution. Then I move onto testing the solution, and finish off with a conclusion on the successes and failures of the solution.

\section{Analysis}

The Epsilon source is broken into many well-organised packages. The packages \verb+org.eclipse.epsilon.eol.*+ contain all of the code that is specific to the EOL language, and so these will be the primary focus of this analysis.

The package \verb+org.eclipse.epsilon.eol.execute+ unsurprisingly contains the code to execute an EOL program. To perform statement coverage, it is necessary to determine which statements have been executed. 

Some analysis of the execute package and its sub-packages has uncovered the interface \verb+IExecutionListener+, as shown in Figure \ref{lst:IExecutionListener}. An instance of a class that implements this interface can be added to a list of execution listeners. When a statement is about to execute, each object in the list has its \verb+aboutToExecute+ method called, and similarly after each statement has executed, each object in the list has its \verb+finishedExecuting+ method called.

\begin{figure}
	\lstinputlisting[language=java]{code/IExecutionListener.java}
	\caption{The public interface IExecutionListener}
	\label{lst:IExecutionListener}
\end{figure}

In the literature review the purpose of an Abstract Syntax Tree (AST) was described. The first parameter of both methods in \verb+IExecutionListener+ is an abstract syntax tree object. The Abstract Syntax Tree class in Epsilon is designed in such a way that each vertex is an object of type AST, and each vertex has a list of children vertices, as well as a pointer back to the parent vertex. The parent vertex will have a null pointer in place of a pointer to a parent vertex, and leaf of the tree will have an empty list of children.

The AST object that is passed as a parameter into both functions is a pointer to the vertex in the program's abstract syntax tree that is about to be executed or has just been executed, depending on the method being called.
