\chapter{Branch Coverage}

\section{Introduction}
This chapter details my effort to implement branch coverage. The structure of the chapter is largely the same as the previous chapter. However, there is a lot more detail provided on the algorithm as it is thought to be the only detailed description on the process of converting an Abstract Syntax Tree to a Control Flow Graph.

\section{Analysis}

As detailed in the literature review, branch coverage is how many conditional statements have had all possible paths executed. So for an \verb|if| statement, if it only ever evaluates to \verb|true| then the branch coverage at that vertex is 50\%. This becomes a problem when you have code such as in Figure \ref{lst:branchingExample}.If the \verb|if| statement always evaluates to true during testing, statement coverage will show as being over 99\%. However, if it ever evaluates to false then \verb|someObject| won't be initialised, and an exception will be thrown later on if somewhere else \verb|someObject| is referenced.

\begin{figure}
\centering
\begin{minipage}{.33\textwidth}
  \centering
  \lstinputlisting[language=EOL]{code/branch_example.java}
  %\caption{Some sample pseudocode}
  \label{lst:branchingExample}
\end{minipage}%
\begin{minipage}{.5\textwidth}
  \centering
  \includegraphics[scale=0.5]{figures/branchSampleAST.pdf}
  %\caption{The AST of the program in Figure \ref{lst:branchingExample}}
  \label{fig:branchExampleAST}
\end{minipage}
\caption{On the left is some sample pseudocode, and on the right is the AST for the sample pseudocode}
\end{figure}

Branch coverage can counter this by looking at how many of the possible paths after all conditional statements have been executed. So in the code in Figure \ref{lst:branchingExample}, only 50\% of possible paths from that \verb|if| statement have been executed, and so the branch coverage is 50\%.

By looking at the AST (Figure \ref{fig:branchExampleAST}) of the sample code, it would appear that by counting the number of blocks below the \verb|if| vertex, we could determine how many branches in the code there are, and after execution we could see how many of those branches have been executed.

Unfortunately this approach is not perfect. The blocks only appear when curly braces are used. If just a single statement is placed under the \verb|if| statement, then the block is skipped and just a vertex for the single statement appears. So this means that the code is now more complex than it was previously thought to be. Furthermore, \verb|if| statements can have children that never need to be executed (because they just contain information about the conditional), and so my algorithm would need to include details of this. If these were the only drawbacks then I would still choose this approach. However, this kind of caveat occurs for many conditional statements, and so the code that would be produced would be rather unwieldy and difficult to maintain.

The approach therefore that I have chosen to take is actually quite difficult to justify. I suggest that the AST be converted to a Control Flow Graph (CFG), at which point the branches from each vertex will be clear. A record will be made on which edges between vertices have been executed, and the total number of edges will be counted. The edges that have not been recorded will map to the branches that were not taken. The reason that this is difficult to justify is that an extensive search has not come up with any explicit instructions on how to go about generating a control flow graph from an abstract syntax tree. Furthermore, the complexities of special code for each type of statement will still apply when performing the conversion. However, some forward thinking means that the conversion from AST to CFG will be necessary when performing the path coverage because of the formula detailed in the literature review's path coverage section to calculate cyclomatic complexity, and so this effort will solve two problems, and will have a quicker overall development time.

Before beginning development, a list was made of the statements that need to be included. This was done by going through the Epsilon book \citep{epsilonBook} which is a complete source of EOL syntax, but also as well by going through the EuGENia source to see which statements are actually used in real EOL code. The list was then loosely ordered in priority based on the number of uses within the EuGENia source. The list as as follows:

\begin{enumerate}[nolistsep]
\item block
\item if
\item if .. else
\item for
\item while
\item switch
	\begin{enumerate}
	\item case
	\item default
	\item continue
	\end{enumerate}
\item operation
\item return
\item break
\item breakAll
\item continue
\end{enumerate}

For each of the identified statements, I will individually analyse how they can be converted from an AST to a CFG. For each statement a sample AST will be shown, as well as the desired CFG.

\subsection{Start and End}
As discussed in the literature review, a CFG start with a START vertex, and ends with an END vertex. In all the examples below of a CFG, these vertices are present. Each example represents a small subsection of an EOL program. When viewing the examples, the START vertex could be imagined to be where any part of a larger program joins up to the example CFG, and similarly the END vertex can be imagined to be the continue point of the program, where the statements following the example would join up to.

\subsection{The Block}
Block is not actually a statement, but refers to a block of statements. Within a block can be any other set of statements, including other blocks. The contents of a block are often contained within \{ \} braces, but not always (see the case statement).

The block is not conditional in any way. It can have a number of children, which are executed in order of first child (left-most) to last child (right-most). During the conversion of AST to CFG, when a block is encountered it should simply be a case of joining a block vertex from the last statement that was encountered, and joining it to the the block's first child.

The code in Figure \ref{fig:block} is a whole EOL program. The whole program is inside a block statement, as can be seen by the AST in the same figure. There are two statements in the block, and in the AST each statement is represented as a child. In the CFG, each of the children are represented sequentially, so the first child is represented after the block, and then the second child after the first child.

The block could be left out of the CFG, as arguably it does not provide any more information about control flow. For the time being this will be ignored, but at a later stage I will discuss and decide on this. For the rest of this analysis, the block statement may be used to represent any subset of vertices that does not add to the analysis. The input to the block vertex will represent input to the first vertex of the subset, and the output from the block will represent the output from any possible exit vertices from the subset.

\begin{figure}
\centering
\begin{minipage}{.3\textwidth}
  \centering
  \lstinputlisting[language=EOL]{code/statements/block.java}
  %\caption{}
  %\label{lst:blockStatement}
\end{minipage}%
\begin{minipage}{.3\textwidth}
  \centering
  \includegraphics[width=\linewidth]{figures/statements/block_AST.pdf}
   % \caption{}
  %\label{fig:blockAST}
\end{minipage}
\begin{minipage}{.3\textwidth}
  \centering
  \includedot[scale=0.3]{figures/statements/block_CFG}
   % \caption{}
  %\label{fig:blockCFG}
\end{minipage}
\caption{From left to right: The block's code, AST and desired CFG}
\label{fig:block}
\end{figure}

\subsection{The if statement}

The if statement is going to be treated as a separate statement to the if .. else statement, because when it comes to designing and implementing the algorithm, different code will be required for each.

The if statement evaluates its parameter, and if it evaluates to true then it executes the code in the parenthesis that follow it, or if there are no parenthesis, it executes the statement that directly follows it. While the outcome of execution is the same (when there is only one statement executed following an if statement that evaluates to true), the use of parenthesis following the if statement changes the structure of the AST in a way that must be considered during the conversion. Figure \ref{fig:if} shows the if statement that uses parenthesis, and figure \ref{fig:ifnoparen} shows the statement that does not use parenthesis.

When the parenthesis are used, if the if statement evaluates to false then the statement that immediately follows the block must be the next statement to be executed. This will be the sibling of the if statement in the AST. If the if statement evaluates to true, then once it has finished executing the block it must move on to the next statement after the block, which again will be the next sibling of the if statement in the AST. This of course is assuming that a statement within either of the blocks does not divert program flow away from the next statement. A simple example would be to imagine that the contents of the block shown in Figure \ref{fig:if} are modified to include a \verb|return| statement. This is not uncommon in a program, and so must be considered. In this case the the final statement will not be executed, because return will end the current program (or sub-routine).

When no parenthesis are used, the same rules almost apply. Except that rather than looking for the next statement after the block, it will be the second statement following the if statement. This will still be represented in the AST as the next sibling of the if statement.

Another consideration must be made about program flow. Within the conditional part of the if statement, any number of subroutines can exist. It could be viewed as another block of code, and this is why it appears as the first child of the if statement in the AST. For the purpose of control flow, this will be ignored. Any code within the conditional will not modify the flow of the program, as it should only evaluate to either true or false.

\begin{figure}
\centering
\begin{minipage}{.3\textwidth}
  \centering
  \lstinputlisting[language=EOL]{code/statements/if.java}
  %\caption{}
  %\label{lst:blockStatement}
\end{minipage}%
\begin{minipage}{.3\textwidth}
  \centering
  \includedot[width=\linewidth]{figures/statements/if_AST}
   % \caption{}
  %\label{fig:blockAST}
\end{minipage}
\begin{minipage}{.3\textwidth}
  \centering
  \includedot[scale=0.3]{figures/statements/if_CFG}
   % \caption{}
  %\label{fig:blockCFG}
\end{minipage}
\caption{From left to right: The if statement with parenthesis' code, AST and desired CFG}
\label{fig:if}
\end{figure}

\begin{figure}
\centering
\begin{minipage}{.3\textwidth}
  \centering
  \lstinputlisting[language=EOL]{code/statements/ifnoparen.java}
  %\caption{}
  %\label{lst:blockStatement}
\end{minipage}%
\begin{minipage}{.3\textwidth}
  \centering
  \includedot[width=\linewidth]{figures/statements/ifnoparen_AST}
   % \caption{}
  %\label{fig:blockAST}
\end{minipage}
\begin{minipage}{.3\textwidth}
  \centering
  \includedot[scale=0.3]{figures/statements/ifnoparen_CFG}
   % \caption{}
  %\label{fig:blockCFG}
\end{minipage}
\caption{From left to right: The if statement without parenthesis' code, AST and desired CFG}
\label{fig:ifnoparen}
\end{figure}

\subsection{The if .. else statement}

The if .. else statement differs from the if statement in two ways. The first is that control flow will always go down one of two paths (see Figure \ref{fig:ifelse}), rather than potentially going down one or skipping around it. Secondly, the if statement in the AST now has three children rather than just two. The additional child is the block (or single statement if no parenthesis are used) for the else statement.

When implementing the conversion a check will need to be included to see how many children the if statement has. While the EOL language defines an else statement, it is never featured in the AST that is generated and therefore cannot be used to distinguish between an if statement and an if .. else statement. 

Because control flow is forced down one of two paths, the final statement of both paths will now need to link to the next statement that follows the whole if .. else statement. As with the if statement, this makes the assumption that neither of the if or else blocks divert control flow. If they do, then the CFG will need to show this accordingly.

\begin{figure}
\centering
\begin{minipage}{.3\textwidth}
  \centering
  \lstinputlisting[language=EOL]{code/statements/ifelse.java}
  %\caption{}
  %\label{lst:blockStatement}
\end{minipage}%
\begin{minipage}{.3\textwidth}
  \centering
  \includedot[width=\linewidth]{figures/statements/ifelse_AST}
   % \caption{}
  %\label{fig:blockAST}
\end{minipage}
\begin{minipage}{.3\textwidth}
  \centering
  \includedot[scale=0.3]{figures/statements/ifelse_CFG}
   % \caption{}
  %\label{fig:blockCFG}
\end{minipage}
\caption{From left to right: The if .. else statement without parenthesis' code, AST and desired CFG}
\label{fig:ifelse}
\end{figure}

\subsection{The for loop}

\begin{figure}
\centering
\begin{minipage}{.3\textwidth}
  \centering
  \lstinputlisting[breaklines=true,language=EOL]{code/statements/for.java}
  %\caption{}
  %\label{lst:blockStatement}
\end{minipage}%
\begin{minipage}{.3\textwidth}
  \centering
  \includedot[width=\linewidth]{figures/statements/for_AST}
   % \caption{}
  %\label{fig:blockAST}
\end{minipage}
\begin{minipage}{.3\textwidth}
  \centering
  \includedot[scale=0.3]{figures/statements/for_CFG}
   % \caption{}
  %\label{fig:blockCFG}
\end{minipage}
\caption{From left to right: The for loop code (taken from the Epsilon Book \cite{epsilonBook}), AST and desired CFG}
\label{fig:for}
\end{figure}

The for loop in EOL works in the same way as it does in the majority of languages. It iterates over a collection of items, executing the block of code underneath it once for each element in the collection specified.

To represent the for loop in a CFG, there will initially be a for vertex. Then from the for vertex there will be the contents of the block underneath it represented (which is the third child in the example AST in Figure \ref{fig:for}).

A decision that must be made is whether the final vertex of the for loop always returns to the initial for vertex, or whether it has one edge returning to the for loop, and another edge continuing to the next statement after the for loop. For non-final iterations of the loop, the final statement within the for block will always go back to the for loop, but on the final iteration it could continue to the next part of the program. The two options are shown in Figure \ref{fig:forOptions}. The assumption that the block can be multiple statements has to be made, but you can see the two options that are available.

After some consideration (and discussion with my supervisor) I have decided to go for the option that is shown on the left in Figure \ref{fig:forOptions}. My argument is that returning to the for loop after the final iteration more accurately reflects what happens, because the program will return to the for loop to check if there are any more iterations to perform, and if not, continue on to the next statement after the for loop.

Once again statements that alter the control flow must be considered. As with the if statement, if a \verb|return| can be called within the for loop, then this must be reflected in the CFG that is generated. There are also some additional statements that must be considered. These are: \verb|break|, \verb|breakAll| and \verb|continue|. However, these will be discussed later.

Also to be considered with the for loop is the possibility of multiple final statements within the for statement's block. In the example in Figure \ref{fig:for}, there is an if statement that may or may not execute. If it does execute, then the final statement is the final statement within the if block. However, if it does not executed, the final statement is the if statement, which is why I have added an edge from the if vertex back to the for vertex. The actual implementation of this may prove to be difficult, but this will be investigated at a later time.

Up to this point, each statement has only been considered on its own. But the example in Figure \ref{fig:for} actually has an if statement within a for loop. It would be a very easy problem to solve if nested statements weren't allowed, but unfortunately it would make programs very difficult to write. Therefore the algorithm that I implement must be able to deal with any number of different types of nested statements, at any level of depth.

\begin{figure}
\centering
\begin{minipage}{.3\textwidth}
  \centering
  \includedot[width=\linewidth]{figures/for_cfg_1}
   % \caption{}
  %\label{fig:blockAST}
\end{minipage}
\begin{minipage}{.3\textwidth}
  \centering
  \includedot[scale=0.3]{figures/for_cfg_2}
   % \caption{}
  %\label{fig:blockCFG}
\end{minipage}
\caption{The two options for the for loop}
\label{fig:forOptions}
\end{figure}

\subsection{The while loop}

The for loop iterates over a collection of objects, but the while loop iterates as long as its conditional statements evaluate to true. This is the case in most languages, and is the case in EOL. While there are some internal differences between the for and while loops in EOL (as detailed in the Epsilon Book \citep{epsilonBook}), in terms of control flow they are identical. The AST differs slightly because a for loop has 3 children, whereas a while loop only has 2. However, for control flow the only child of interest is the last one for each loop, as this is the block of statements that are executed.

Notice that the CFG example in Figure \ref{fig:while} always returns to the while vertex before continuing the program. This is the same as was decided in the for loop.

When it comes to designing the algorithm to perform the conversion, it will make sense to consider the for and while loops as the same thing, with the minor exception of the child that contains the block of statements being in a different position for each.

\begin{figure}
\centering
\begin{minipage}{.3\textwidth}
  \centering
  \lstinputlisting[breaklines=true,language=EOL]{code/statements/while.java}
  %\caption{}
  %\label{lst:blockStatement}
\end{minipage}%
\begin{minipage}{.3\textwidth}
  \centering
  \includedot[width=\linewidth]{figures/statements/while_AST}
   % \caption{}
  %\label{fig:blockAST}
\end{minipage}
\begin{minipage}{.3\textwidth}
  \centering
  \includedot[scale=0.3]{figures/statements/while_CFG}
   % \caption{}
  %\label{fig:blockCFG}
\end{minipage}
\caption{From left to right: The while loop code (taken from the Epsilon Book \cite{epsilonBook}), AST and desired CFG}
\label{fig:while}
\end{figure}

\subsection{The switch statement}

The switch statement is probably one of the most complex statements in EOL in terms of control flow. The EOL switch case is similar to the Pascal switch statement. The switch statement takes a value as an argument, and then the case statements within the switch statement's block are examined. If one is matched, then the block that follows that case statement is executed, and control is passed to the statement that follows the block of the switch statement.

This differs from the Java switch statement, where every single case statement is examined, and if it matches the switch statement's parameter, then it is executed. This allows for multiple case statement blocks to be executed potentially in a single case statement.

At this point, it sounds like EOL's switch statement is easier for generating a CFG, because coming out of a switch vertex is a link to each of the case vertices, and one that bypasses to the next vertex after the switch block (in case none of the case statements are executed). However, I now look at the continue statement.

\begin{figure}[h]
\lstinputlisting[breaklines=true,language=EOL]{code/switch_example.java}
\caption{An example of a switch statement with fallthrough}
\label{lst:switchFallthrough}
\end{figure}

If a case statement ends with continue in EOL, then the contents of every case block that follow the current case block will be executed. So in the code listing in Figure \ref{lst:switchFallthrough}, because case 0 is executed, and case 0's block finishes with the continue statement, the actual output will be:

\begin{verbatim}
Zero
One
Two
\end{verbatim}

So unlike Java's fallthrough that just checks every other case statement, it actually executes every following case statement, regardless of the case values. This complicates control flow quite a bit, because if a continue statement is contained in any of the case statements, then edges have to be added from the end that case statement and every subsequent case statement block to the start of the following case statement block.

As well as case, switch can also have the default statement within its block. The default statement is always after all of the case statement, and is executed when none of the case statements were executed (or when a continue statement was called in one of the case statements). In terms of program flow, this will remove the edge from the switch vertex to the statement following the switch block, because the default statement forces the execution of something within the switch block. As with all of the case statements, the final statement of the default block moves on the execute the next statement after the switch block.

The switch statement requires two examples to fully demonstrate possible flow. Figure \ref{fig:switch1} shows a switch statement that does not have a default statement. That means that it is possible for none of the case statements to be executed, and so there should be an edge from the switch vertex straight to the statement that follows the switch block (which in this case happens to be the end vertex because there are no more statements following the switch block). Then notice that the continue statement on the left-most case joins to the block of the next case statement, and that a vertex has been added from the end of the second case statement to the block of the third case statement.

\begin{figure}
\centering
\begin{minipage}{.6\textwidth}
  \centering
  \lstinputlisting[breaklines=true,language=EOL]{code/statements/switch1.java}
    \includedot[width=\linewidth]{figures/statements/switch1_AST}
  %\caption{}
  %\label{lst:blockStatement}
\end{minipage}%
\begin{minipage}{.3\textwidth}
  \centering
  \includedot[scale=0.3]{figures/statements/switch1_CFG}
   % \caption{}
  %\label{fig:blockCFG}
\end{minipage}
\caption{Top left: The example switch code that includes a continue statement. Bottom left: The AST of the example switch code. Right: The desired CFG for this example. Notice the continue statement on the left, and that there is an edge from the switch statement straight to the end statement.}
\label{fig:switch1}
\end{figure}

\begin{figure}
\centering
\begin{minipage}{.3\textwidth}
  \centering
  \lstinputlisting[breaklines=true,language=EOL]{code/statements/switch2.java}
  %\caption{}
  %\label{lst:blockStatement}
\end{minipage}%
\begin{minipage}{.3\textwidth}
  \centering
  \includedot[width=\linewidth]{figures/statements/switch2_AST}
   % \caption{}
  %\label{fig:blockAST}
\end{minipage}
\begin{minipage}{.3\textwidth}
  \centering
  \includedot[scale=0.3]{figures/statements/switch2_CFG}
   % \caption{}
  %\label{fig:blockCFG}
\end{minipage}
\caption{From left to right: The example switch statement that includes a default case, the AST for the example, and the desired CFG. Notice in the CFG that there is no edge from the switch vertex to the end vertex because there is a default vertex.}
\label{fig:switch2}
\end{figure}
